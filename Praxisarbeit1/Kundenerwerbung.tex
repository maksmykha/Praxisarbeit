\chapter{Phase 1: Kundenerwerbung} 
Auf dem Markt existiert eine vielfältige Anzahl von IT-Beratungsunternehmen, die von kleinen Boutique-Firmen bis hin zu großen Unternehmen wie Capgemini reichen. Alle Anbieter verfolgen das Ziel, Kunden zu gewinnen, und nutzen dafür unterschiedliche Ansätze, beispielsweise attraktive Verträge mit geringen Kosten und minimalem Risiko für die Kunden oder herausragende Präsentationen ihrer Arbeit mithilfe von Demonstratoren\cite{muhonen2013qualification}.
\section{Ausgangslage und Aufgabenstallung} %Ich beschäftige mich mit der Showcase für die Kunden um das Projekt zu kriegen.
Capgemini ist ein großes Unternehmen mit verschiedenen Einheiten, die, wie bereits beschrieben, teilweise ähnliche Tätigkeiten ausführen. Die Abteilung "Product Lifecycle Management/Digital Manufacturing" von Capgemini \cite{Capgemini} verfügt über einige PLM-Demonstratoren. Ähnliche Demonstratoren existieren jedoch auch in anderen Einheiten, über die die zentrale Abteilung nur begrenzt informiert ist.\newline
Das Ziel des Managements ist es, eine zentrale Übersicht über alle vorhandenen PLM-Demonstratoren zu erstellen, damit alle Mitarbeitenden darauf zugreifen können. Dies soll nicht nur die Transparenz innerhalb des Unternehmens erhöhen, sondern auch die Kommunikation und Zusammenarbeit zwischen den verschiedenen Einheiten stärken.
\section{Anforderungen für die Erstellung der Demonstratorenübersicht} %welche versionen die Demonstratoren haben, Ziel des Demonstrators, Beschreibungen etc.
Capgemini Deutschland vefügt über eine Vielzahl von Demonstratoren. Es soll eine toolgestützte Wiki-Übersicht erstellt werden, die verschiedene Analysemöglichkeiten bietet, wie etwa die Sortierung und Filterung der PLM-Demonstratoren nach Kriterien wie Vendoren oder Versionen.
Zudem ist sicherzustellen, dass die Seite leicht auffindbar ist. Dafür sollte sie auf einer zentralen Plattform integriert werden, die allen Mitarbeitenden zugänglich ist und eine intuitive Navigation gewährleistet.
\section{Tools-Analyse}%Welche Tools gibt und welche sind geeignet für meinen Demostrator(Beschränung von Unternehmen betrachten)
Es existiertz zahlreiche Tools, um eine solche Übersicht zu erstellen, darunter Wiki.js, Jira Confluence, Microsoft SharePoint, Zendesk, ClickUp und viele mehr. Allerdings verwendet Capgemini in der Regel die Produkte von Atlassian und Microsoft . Aus diesem Grund wird die Priorität auf Confluence\cite{Confluence} und SharePoint\cite{SharePoint} gelegt, um die Demonstratorenübersicht zu entwickeln. Beide Tools sind starke Tools mit verschiedenen zwar Focus.
\newpage
\begin{longtable}{|p{4.5cm}|p{5.5cm}|p{5.5cm}|}
	\hline
	\textbf{Kriterium} & \textbf{Confluence} & \textbf{SharePoint} \\ \hline
	\textbf{Wiki-Funktionalität} & Entwickelt als kollaborative Plattform für Wissensmanagement und Dokumentation & Kann als Wiki verwendet werden, aber primär für Dokumentenmanagement konzipiert \\ \hline
	\textbf{Datenintegration} & Nahtlose Integration mit Jira und anderen Atlassian-Produkten sowie Microsoft-Teams & Enge Integration mit Microsoft-Produkten, wie Power BI, PowerPoint usw..\\ \hline
	\textbf{Benutztfreundlichkeit} & Intuitiv und flexibel, ideal für Teams, die Inhalte gemeinsam bearbeiten & Komplexere Einrichtung, erfordert mehr Anpassung für Wiki-Zwecke\\ \hline
	\textbf{Suchenfunktionen} &starke Suchfunktionen mit Tags und Kategorien & Gute Suchfunktionen, aber weniger intuitiv für Wiki-Inhalte. Möglichkeit leicht die Seite durch internes Internet finden  \\ \hline
	\textbf{Kollaboration} &Echtzeit-Bearbeitung von Seiten und Kommentaren & wenig dynamisch, eher wie "branching" in GitHub \\ \hline
	\caption{Vergleich zwischen Jira Confluence und Micorosoft SharePoint}
\end{longtable}
Einerseits ist es von hoher Bedeutung, dass die geplante Seite über umfassende Analysemöglichkeiten verfügt, was durch den Einsatz von Confluence effizient umgesetzt werden kann. Andererseits ist die Auffindbarkeit der Seite ebenso entscheidend. Angesichts der bestehenden Infrastruktur bei Capgemini ist es naheliegend, zwei SharePoint-Plattformen miteinander zu verknüpfen, da dies technisch leicht umsetzbar ist. Danach kann die Seite mit der Übersicht schnell gefunden werden. Darüber hinaus bietet SharePoint eine enge Integration mit Microsoft-365-Diensten wie Power BI, wodurch ebenfalls erweiterte Analysemöglichkeiten sich schaffen lassen.
\section{Erstellung der Demonstratorenübersicht}
Zunächst sollte eine zentrale Seite erstellt werden, die eine Einführung und Erklärung der Übersicht bereitstellt. Diese Seite dient als Einstiegspunkt und verweist auf die spezifischen Seiten der verschiedenen PLM-Vendoren mit ausführlichen Informationen über Demostratoren. Die ausführliche Information über jeden Demostrator besteht aus Abteilung, Kontaktpersonen, Fähigkeiten, Kompatibilität, zusätzliche Dokumentation usw.. Für die Umsetzung dieser strukturierten Informationsübersicht eignet sich das benutzerfreundliche Tool Microsoft Lists.\cite{List} Durch die Integration in die Microsoft-Umgebung ist zudem eine einfache Kollaboration und Pflege der Daten gewährleistet.\newline
Ergänzend dazu kann Power BI \cite{Power_BI} die strukturierte Übersicht analysieren und umfangreiche Filtermöglichkeiten für Demonstratoren bieten. Dies unterstützt besonders bei der gezielten Suche und erleichtert die Navigation durch die Daten.
Einige Demonstratoren bilden nur einen bestimmten Teil des gesamten E2E-PLM-Prozesses ab, doch besteht die Möglichkeit, dass sich mehrere Demonstratoren fusionieren lassen, um einen größeren Demonstrator zu schaffen. Diese Zusammenführung kann mithilfe der Ja/Nein-Funktion in Microsoft Lists gesteuert werden, wobei kann in jeder Phase des PLM-Prozesses eindeutig gekennzeichnet werden, ob ein Demonstrator die jeweilige Phase vorstellt.